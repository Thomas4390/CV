%-------------------------------------------------------------------------------
%	SECTION TITLE
%-------------------------------------------------------------------------------
\cvsection{Formation}


%-------------------------------------------------------------------------------
%	CONTENT
%-------------------------------------------------------------------------------
\begin{cventries}

%---------------------------------------------------------
  \cventry
    {\href{https://www.hec.ca/programmes/maitrises/maitrise-ingenierie-financiere/}{M.Sc. en ingénierie financière - Mémoire}}
    {\href{https://www.hec.ca/}{HEC Montréal}}
    {Montréal, Canada}
    {Août 2022 - Décembre 2024}
    {
      \begin{cvitems}
        \item{\textbf{Mémoire:} Gamma Exposure and Intraday Volatility Dynamics in the SPX Options Market -- Analyse de l'impact du gamma hedging des teneurs de marché sur la volatilité réalisée à partir de données d'options aux 5 minutes (2B+ points de données, août 2020--juil. 2025).}
        \item{\textbf{Moyenne:} 3.88/4.3. Récipiendaire de la bourse Deloitte M.Sc. et de la bourse d'excellence Jocelyne \& Jean C. Monty pour le mérite académique.}
        \item{\textbf{Hackathons:} Équipe gagnante du défi Banque Nationale au ConUHacks VIII 2024 (1000+ participants). Top 10 finaliste au McGill FIAM Portfolio Management Hackathon (66 équipes).}
        \item{\textbf{Leadership:} Vice-président Recherche au Club de Trading -- organisation d'ateliers sur les produits dérivés et le trading systématique. Analyste R\&D au Fonds d'investissement BNI-HEC (5M\$ d'actifs) -- recherche sur les actions et analyse de portefeuille.}
        \item{\textbf{Cours:} Calcul stochastique, Évaluation des produits dérivés, Titres à revenu fixe, Économétrie financière, Apprentissage automatique en finance, Méthodes numériques, Gestion des risques.}
      \end{cvitems}
    }

  \cventry
    {\href{https://www.hec.ca/etudiants/mon-programme/baa/specialisations/economie-appliquee-finance-mathematiques.html}{B.A.A. | spécialisation en Mathématiques, Finance et Économie}}
    {\href{https://www.hec.ca/}{HEC Montréal}}
    {Montréal, Canada}
    {Août 2017 - Mai 2022}
    {
      \begin{cvitems}
        \item{\textbf{Moyenne:} 3.8/4.3 pour les cours de spécialisation. Parcours quantitatif sélectif combinant mathématiques avancées, économie et finance.}
        \item{\textbf{Leadership:} Vice-président Éducation au Comité de Science des Données -- organisation d'ateliers Python et séminaires d'apprentissage automatique.}
        \item{\textbf{Compétitions:} Participant à la compétition RITC Trading -- simulation de trading sur les marchés d'actions et de produits dérivés.}
        \item{\textbf{Enseignement:} Auxiliaire d'enseignement en Microéconomie intermédiaire (200+ étudiants). Tuteur privé en mathématiques (Calcul, Algèbre linéaire).}
        \item{\textbf{Cours:} Économétrie, Gestion de portefeuille, Options et contrats à terme, Finance quantitative, Modèles probabilistes, Finance d'entreprise.}
      \end{cvitems}
    }

    \cventry
    {\href{https://admission.umontreal.ca/programmes/mineure-en-mathematiques/}{Mineure en mathématiques}}
    {\href{https://www.umontreal.ca/}{Université de Montréal}}
    {Montréal, Canada}
    {Août 2018 - Août 2022}
    {
      \begin{cvitems}
        \item{Programme complémentaire en mathématiques complété simultanément avec le B.A.A. pour renforcer les fondations quantitatives.}
        \item{\textbf{Cours:} Théorie des probabilités, Statistique mathématique, Processus stochastiques, Algèbre linéaire, Calcul à plusieurs variables, Analyse réelle, Programmation (C/C++).}
      \end{cvitems}
    }

%---------------------------------------------------------
\end{cventries}
